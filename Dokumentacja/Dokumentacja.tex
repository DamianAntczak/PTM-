\documentclass[a4paper,10pt,oneside]{book}
\usepackage{standalone}
\usepackage{suffix}
\usepackage{ifthen}
\usepackage{anysize}
\setlength{\columnsep}{5mm}
\marginsize{2.3cm}{2.3cm}{2cm}{1.7cm}
\usepackage[polish]{babel}
\usepackage[utf8]{inputenc}
\usepackage{listings}
\usepackage{graphicx}
\usepackage{color}
\usepackage{indentfirst}
\usepackage{marvosym}
\usepackage{enumitem}
\usepackage[final]{pdfpages}
\usepackage[]{hyperref}
\usepackage{array}
\usepackage[ampersand]{easylist}
\usepackage[section] {placeins}
\usepackage{listings}
\usepackage{hyperref}
\usepackage{bookmark}
\bookmarksetup
{
  numbered,
  open
}
\usepackage{tabularx,ragged2e}
\usepackage{color}
\usepackage[nodayofweek]{datetime}
\newdateformat{mydate}{\monthname[\THEMONTH] \THEYEAR}
\usepackage{titlesec}
\definecolor{dkgreen}{rgb}{0,0.6,0}
\definecolor{gray}{rgb}{0.5,0.5,0.5}
\definecolor{mauve}{rgb}{0.58,0,0.82}
\pagenumbering{arabic}
\renewcommand{\baselinestretch}{1.15}
\lstset
{
  frame=tb,
  language=C++,
  aboveskip=3mm,
  belowskip=3mm,
  showstringspaces=false,
  columns=flexible,
  basicstyle={\small\ttfamily},
  morekeywords={uint8_t,int8_t,uint16_t,int16_t,uint32_t,int32_t,friend,class,struct,string,int,std,boost,vector,map,list,queue,deque,array, shared_ptr,weak_ptr,nullptr,cout,cin},
  numbers=left,
  numbersep=10pt,
  numberstyle=\tiny\color{mygray},
  stepnumber=2,
  numberstyle=\tiny\color{gray},
  keywordstyle=\color{blue},
  commentstyle=\color{dkgreen},
  stringstyle=\color{mauve},
  breaklines=true,
  breakatwhitespace=true,
  tabsize=3,
}

\makeatletter
\renewcommand*{\thesection}{\arabic{section}}
\renewcommand\section{\leftskip 0pt\@startsection {section}{1}{\z@}%
                                   {-3.5ex \@plus -1ex \@minus -.2ex}%
                                   {2.3ex \@plus.2ex}%
                                   {\normalfont\Large\bfseries}}
\renewcommand\subsection{\leftskip 0pt\@startsection{subsection}{2}{\z@}%
                                     {-3.25ex\@plus -1ex \@minus -.2ex}%
                                     {1.5ex \@plus .2ex}%
                                     {\normalfont\large\bfseries\leftskip 4ex}}
\newcommand\Xsubsubsection{\@startsection{subsubsection}{3}{\z@}%
                                     {-3.25ex\@plus -1ex \@minus -.2ex}%
                                     {1.5ex \@plus .2ex}%
                                     {\normalfont\normalsize\bfseries\leftskip 8ex}}
\renewcommand\subsubsection[1]{\Xsubsubsection{#1}\leftskip 3ex}
\makeatother
\renewcommand{\baselinestretch}{1.15}
\renewcommand\tabularxcolumn[1]{>{\Centering}p{#1}}
\setlist[description]{leftmargin=\parindent,labelindent=\parindent}
\makeatletter
% Save the original \part declaration
\let\old@part\part

% To that definition, add a new special starred version.
\WithSuffix\def\part*{
  % Handle the optional parameter.
  \ifx\next[%
    \let\next\thesis@part@star%
  \else
    \def\next{\thesis@part@star[]}%
  \fi
  \next}

% The actual macro definition.
\def\thesis@part@star[#1]#2{
  \ifthenelse{\equal{#1}{}}
   {% If the first argument isn’t given, default to the second one.
    \def\thesis@part@short{#2}
    % Insert the actual (unnumbered) \part header.
    \old@part*{#2}}
   {% Short name is given.
    \def\thesis@part@short{#1}
    % Insert the actual (unnumbered) \part header with short name.
    \old@part*[#1]{#2}}

  % Last, add the part to the table of contents. Use the short name, if provided.
  \addcontentsline{toc}{part}{\thesis@part@short}
}
\makeatother

\newcommand*{\justifyheading}{\centering}
\titleformat{\chapter}[display]
  {\normalfont\huge\bfseries\justifyheading}{\chaptertitlename\ \thechapter}
  {20pt}{}  
  [
  \rule{8cm}{0.4pt}
  ]
\makeatletter
\@addtoreset{chapter}{part}
\makeatother 

\begin{document}
\begin{sloppypar}

\begin{titlepage}
\begin{center}
% Upper part of the page. The '~' is needed because \\
% only works if a paragraph has started.
\includegraphics[width=0.15\textwidth]{./img/logo.jpg}~\\[1.3cm]
\textsc{\LARGE Politechnika Poznańska}\\[4cm]
\textsc{\Large Projekt zaliczeniowy z Podstaw Techniki Mikroprocesorowej}\\[0.1cm]
% Title
\rule{10cm}{0.4pt}\\[0.4cm]
{ 
	\huge \bfseries AutoControl - Samochodowy minikomputer pokładowy \\[0.35cm] 
	\small Z wykorzystaniem STM32 \\[0.4cm] 
}
\rule{10cm}{0.4pt} \\[3.5cm]
% Author and supervisor
\noindent
\begin{minipage}{0.4\textwidth}
\begin{flushleft} 
\large
\emph{Autorzy:}
\\ Damian Antczak
\\ Szymon Kłebowski 
\\ Michał Pietrzak 
\end{flushleft}
\end{minipage}%
\begin{minipage}{0.4\textwidth}
\begin{flushright} \large
\emph{Prowadzacy:} 
\\mgr~\textsc{Michał Nowicki} ~\\ ~\\ ~\\ ~\\ ~\\ ~\\ 
\end{flushright}
\end{minipage}
\vfill
% Bottom of the page
{\large Czerwiec 2015}
\end{center}
\end{titlepage}

\clearpage
\pagenumbering{arabic}
\tableofcontents

\clearpage
\vfill
\huge

	\addcontentsline{toc}{part}{Prolog}
	\begin{center}
	    \textbf{\\Prolog\\[1.3cm]}
	\end{center}
	\normalsize

	\paragraph{Link do filmiku prezentujacego działanie: }
		\underline{\href{http://google-styleguide.googlecode.com/svn/trunk/cppguide.html}{Kliknij, żeby obejrzeć}}
	\paragraph{Link do GitHub'a: }
		\underline{\href{http://google-styleguide.googlecode.com/svn/trunk/cppguide.html}{Kliknij, żeby otworzyć w przegladarce}}
	\paragraph{Krótki wstep: }
		Tutaj coś krótko opisać.
  
\clearpage
\vfill
\huge

	\addcontentsline{toc}{part}{Troche szczegółów}
	\begin{center}
	    \textbf{\\Troche szczegółów\\[1.3cm]}
	\end{center}
	\normalsize
	\paragraph{Coś tam}
	Coś tam
	
	\paragraph{Coś tam}
	Coś tam
	
	\paragraph{Coś tam}
	Coś tam
	\\[0.7cm]

\clearpage
\vfill
\huge

	\addcontentsline{toc}{part}{Podsumowanie}
	\begin{center}
	    \textbf{\\Podsumowanie\\[1.3cm]}
	\end{center}
	\normalsize
	\paragraph{Coś tam}
	Coś tam
	
	\paragraph{Coś tam}
	Coś tam
	
	\paragraph{Coś tam}
	Coś tam
	\\[0.7cm]
	
\clearpage
%\addcontentsline{toc}{part}{Appendix I  \small{coding style}}
%\part*{Appendix I}
%\includepdf[pages=-]{CodingStyle.pdf}

\end{sloppypar}
\end{document}